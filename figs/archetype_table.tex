\begin{enumerate}
 \item Reflex
\begin{itemize}[label={}]
\item Q5: Is a baby urinating a behavior?
\item Q10: Is a reflex a behavior?
\item Q15: The sucking reflex is when babies instinctively suck anything that touches the roof of their mouth. Is the sucking reflex a behavior?
\item Q26: The knee-jerk reflex is when a tap of a hammer results in the leg extending once before coming to rest. Is the knee-jerk reflex in adults a behavior?
\item Q31: Is an adult urinating a behavior?
\end{itemize}
 \item Actions
\begin{itemize}[label={}]
\item Q1: Is behavior the output of a sequence of non-motor signals (eg thinking)?
\item Q3: A person watches a movie. Is that person producing a behavior?
\item Q9: A person sees food that they like but choose not to move toward it. Is that person behaving?
\item Q11: An animal watches a movie and tries to decide if objects are moving left or right. As it sits there, trying to decide about which way objects are moving, is it behaving?
\item Q13: There are behaviors that you cannot directly observe from possible movements.
\item Q16: Can invertebrates behave?
\item Q18: Is a dog that follows a scent trail behaving?
\item Q24: A behavior is always potentially measurable
\item Q27: Is behavior the output of a sequence of motor signals (eg walking)?
\item Q28: When fish are schooling (moving in a group), is the school (the whole group of fish) behaving?
\item Q33: A person sees a video on TV. Is that person producing a behavior?
\item Q40: Behaviors have magnitude; speaking loudly is not the same behavior as speaking quietly.
\item Q42: A bacteria can sense chemicals in its environment. It will turn when it senses more of one type of chemical, and move forward if it senses less of that chemical. Is this bacteria behaving?
\item Q44: Bacteria will release chemicals that attract other bacteria. The bacteria then move around as a group. Are these bacteria producing behavior?
\item Q46: A person reads a book. Is that person producing a behavior?
\item Q48: Can invertebrates weigh costs and benefits in order to produce behavior?
\end{itemize}
 \item Understanding the mind
\begin{itemize}[label={}]
\item Q8: Behaviors cannot be studied unless anthropomorphized so that the human experimenter can talk about them.
\item Q22: Is a mouse that is held in place while choosing between two options performing the same behavior as a mouse that is free to move around while choosing between the same two options?
\item Q23: Does a behavior need to be intentional (does every behavior an animal produces have a purpose)?
\item Q25: Does a behavior always relate to something (eg, walking toward something versus 'just walking')?
\item Q36: A rat has a dislike for salty food. Is disliking salty food a behavior?
\end{itemize}
 \item Motor or sensorimotor
\begin{itemize}[label={}]
\item Q2: All behaviors an animal performs can potentially be identified from recorded video data by using a smart enough computer algorithm.
\item Q4: There is always a precise time you can point to a behavior starting or ending.
\item Q30: A behavior is always in response to the environment
\item Q32: You are wearing a virtual reality headset. You see a virtual tree and reach out to touch it. Is this the same behavior as reaching out to a real tree?
\item Q35: Behaviors are always discrete; you are either performing that behavior or you are not.
\item Q39: A behavior always involves interaction with the environment ?
\item Q45: A behavior is always the output of motor activity
\end{itemize}
 \item Non-animal
\begin{itemize}[label={}]
\item Q12: A person sweats in response to hot air. Is this person behaving?
\item Q17: A neuron spikes. It produces more spikes when the animal’s head is in a particular direction. Is that neuron producing a behavior?
\item Q19: Can computer programs behave?
\item Q20: Is sweating a behavior?
\item Q37: A thermostat turns on air conditioning in response to heat. Is the thermostat behaving?
\item Q38: Can cells behave?
\item Q43: Particles move around. Are the particles behaving?
\item Q47: Is a sponge filtering water a behavior?
\end{itemize}
 \item Cognition
\begin{itemize}[label={}]
\item Q6: We have to understand what it is like to be a bat to understand its behavior.
\item Q7: Working memory is temporarily holding something in memory. Is working memory a behavior?
\item Q14: An animal hears one sound, then another. In its mind, it compares the two sounds. Is it behaving?
\item Q21: What behavior an animal is producing can only be understood by considering its unique way of sensing and experiencing the world.
\item Q29: Is learning a behavior?
\item Q34: An animal is thirsty. It must choose between two levers, one of which will provide water. Would you refer to the animal seeking relief from thirst as the primary behavior the animal was producing?
\item Q41: An animal reacts defensively to a picture of a predator. Over repeated exposures to this picture, it slowly stops reacting. Is this adaptation a behavior?
\end{itemize}
\end{enumerate}
